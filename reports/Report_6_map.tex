\documentclass{article} 
\usepackage[utf8]{inputenc} 
\title{report 6 : map} 
\author{antoine ranson } 
\date{November 2019} 
\begin{document} 
\maketitle 
\section{How to implement the labwork} 
\subsection{Binarization} 
First you need to set a threshold. For every pixel which value's is superior to the thresold, you modify its value to 255 to set it on white.
On the other hand, if the pixel's value is inferior to the threshold, you set it on 0 so that the pixel appears black. So you just have a simple if-else comparison. 

\subsection{Brightness control} 
For brightness control, you need to more parameters for the kernel : a bool that indicates if you want to increase or decrease brightness, and the percentage of how much you want to change 
it.\newline
Then if you want to increase brightness : you convert the percentage on a 0 to 255 scale, and you add it to each pixel's value. You just need to check that the sum is not superior to 255, because in this case, 
you just set the pixel's value to 255.\newline
 If you want to dicrease brigthness, you need to deduct the converted percentage from the pixel's value, and check that it is not inferior to 0. \subsection{Blending two 
images} To blend image, you need first to get the second image from the \textbf{argv}. To do this, I just add to 
labwork a variable inputImage2, and a function loadInputImage2 which does exactly the same as loadInputImage, 
but on inputImage2.\newline 
Then, you just need for each pixel of the output, to take the average of the two 
corresponding pixels of the two input image. For the RGB image, you take each channel independantly. 

\end{document}
