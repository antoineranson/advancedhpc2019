\documentclass{article} 
\usepackage[utf8]{inputenc} 
\usepackage{amsmath} 
\usepackage{amssymb} 

\title{Report 10 : kuwahara}
\author{antoine ranson } 
\date{November 2019} 
\begin{document} 

\maketitle 

\section{How to implement the labwork ?}

 The kernel, named kuwahara, will take 5 parameters in input : 
\begin{itemize} 
\item the input image in RGB. 
\item the V's array from the image converted into HSV. 
\item the wanted output. 
\item the input image's width and height (these are unnecessary because we have the input image, but it is clearer this way). \end{itemize} 


Then, inside the kernel :\newline 
We need to define the size of the windows used in the kuwahara filter : $\omega$. We will also need variables to calculate the mean, itself needed to calculate the standard deviation. 
One of it for each window, so 4 for the mean, 4 for the standard deviation. The standard deviations of each 
window will be stored in an array of size 4, so that finding the minimum of these SD is easy. We also need a 
mean for the R, the G and the B of the window with the smallest standard deviation. To do so, we would rather 
use an array of size 3. \newline 

Then, here are the computations : 
\begin{enumerate} 
\item We calculate the mean of the V array for each window. 
\item We calculate the SD of each window with the corresponding formula : $ \sigma^2 = \frac{\displaystyle\sum_{i=1}^{n}(V(i) - \mu)^2 }{n} $ , where $\sigma$ is the SD and $\mu$ is the 
mean. Here there is no need to take the square root, because we use the SDs (which are positive) to get a 
minimum, and the function power of two keeps the comparison sens on $\mathbb{R}^{*}$.
\item We get the miminum of the standard deviations : this gives us the index of the window. 
\item We calculate the average value for each channel R,G and B only in the window we identified previously. 
\item Then we affect this average value to the corresponding window of the output. 
\item To deal we the edges, we just write the value of the pixel's input into the output. 

\end{enumerate} 

We have no performance optimizations, so there is no speedup to measure.

\end{document}
