\documentclass{article} 
\usepackage[utf8]{inputenc} 
\title{report 5 : Gaussian Blur} 
\author{antoine ranson } 
\date{November 2019} 
\begin{document} 

\maketitle 

\section{How to implement the gaussian blur filter} To implement the gaussian blur filter, we will use one kernel. In this kernel, we first declare the following gaussian matrix of size 7x7 we will use to filter the image: \newline
 
\[
   M=
  \left[ {\begin{array}{ccccccc}
   0 & 0 & 1 & 2 & 1 & 0 & 0 \\
   0 & 3 & 13 & 22 & 13 & 3 & 0 \\
   1 & 13 & 59 & 97 & 50 & 13 & 1\\
   2 & 22 & 97 & 159 & 97 & 22 & 2\\
   1 & 13 & 59 & 97 & 50 & 13 & 1\\
   0 & 3 & 13 & 22 & 13 & 3 & 0 \\
   0 & 0 & 1 & 2 & 1 & 0 & 0 \\
  \end{array} } \right] \] \newline 
To avoid using negative numbers, I chose to manipulate the tidx and tidy by 
substracting 3 of each, o that the calcuation would begin in the upper-left corner, and not in the middle of the 
matrix. Then, if there is a shared memory, we need to declare three tiles in the shared memory, one for each 
color's channel (R,G and B). 

\subsection{With the shared Memory} 
Before any calculation, we need to fill the 
declared tiles. To do so, we just write the current pixel at its thread index (x,y). Each color's channel works 
the same, so we will only focus on one channel next. We now need to check that the current thread does not 
overflow the boundaries of the tile, so that its value makes sens. If there is no problem, we loop on 2 
dimension to perform the 2D convolution, which consists in the accumulation of the corresponding matrix value 
$m_{i,j}$ times the corresponding pixel value $p_{i,j} $ : 
$p_{k,l} = \displaystyle\frac{1}{\displaystyle\sum_{p,q\in [0...6]^2} m_{p,q}} \times \displaystyle\sum_{i,j\in [0...6]^2} m_{i,j}\times p_{i,j}$, 
where $p_{k,l}$ is the new pixel's value. 

Here, we are not taking care of the edges, which cause the small abnormalities in the display of the image.
\end{document}
