\documentclass{article} 
\usepackage[utf8]{inputenc} 
\title{Report 7 : reduce} 
\author{antoine ranson } 
\date{November 2019} 
\begin{document} 

\maketitle 

\section{How to implement the labwork ?} \newline 
\begin{itemize} 
\item Declare a new kernel grayscale\_stretch which have as parameters : the input image, the output image, and the image width. 
\item In this kernel, reference a share memory space of the corresponding block size. 
\item In the shared memory, copy the block image data. 
\item Wait for all the threads to synchronize. 
\item Then to find the maximum :
\indent \begin{enumerate} 
\item compare the pixels two by two, and store the biggest in the current position of the shared memory. 
\item wait for the threads to synchronize. 
\item Do it all over again with the superior power of two, until you reach the block width. 
\item from the thread 0, store the maximum, which is in the first element of the shared memory, in 
a local variable . 
\end{enumerate} 
\item To find the minimum, do the same as previous but storing the smallest pixel for each comparison. 
\item Finally, for each output pixel g', you apply the following formula : 
$\frac{255\times(g - min)}{max - min}$ where g is the input image corresponding pixel. 
\end{itemize} 

We have no performance optimization, so there is no speedup to measure. 

\end{document}
