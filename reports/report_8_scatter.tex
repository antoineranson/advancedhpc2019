\documentclass{article} 
\usepackage[utf8]{inputenc} 
\usepackage{amsmath} 
\title{report 8 :scatter} 
\author{antoine ranson } 
\date{November 2019} 
\begin{document} 
\maketitle 
\section{How to implement the labwork ?} 

First of all, in the device code, we need to declare 3 array of double which will be the H,S and V array. 
Then, we declare the 2 kernels RGB2HSV and HSV2RGB. 
\begin{itemize}
    \item In RGB2HSV :\newline
    We store the R,G and B value of the pixel in an array of size 3. We then use this array to find the minimum 
and maximum among the 3 channels.\newline
    Once this is done, we just apply the following formulas :
    \begin{equation}
        H =
        \begin{cases}
        0 & \text{si } \Delta =0 \\
        60\times(\frac{G-B}{\Delta} \text{ mod } 6) & \text{si max = R}\\
        60\times(\frac{B-R}{\Delta} +2) & \text{si max = G}\\
        60\times(\frac{R-G}{\Delta} +4) & \text{si max = B}\\
        \end{cases}
        \end{equation}
        \begin{equation}
        S =
        \begin{cases}
        O & \text{si max = 0}\\
        (\frac{\Delta}{\text{max}}) & \text{si } max \ne 0\\
        \end{cases}
        \end{equation}
        \newline
        \begin{equation}
            \text{ V = max}
        \end{equation}
        
    \item In HSV2RGB :\newline
    We just apply the formulas given in the slides to convert HSV to RGB.
    
\end{itemize}
   
 There is no remarkable differences between the input image and the output image which results of the 
conversion into HSV and back into RGB. \newline\newline 

We have no performance optimization, so there is no 
speedup to measure.

\end{document}
